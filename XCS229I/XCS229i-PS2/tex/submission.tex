% This contents of this file will be inserted into the _Solutions version of the
% output tex document.  Here's an example:

% If assignment with subquestion (1.a) requires a written response, you will
% find the following flag within this document: <SCPD_SUBMISSION_TAG>_1a
% In this example, you would insert the LaTeX for your solution to (1.a) between
% the <SCPD_SUBMISSION_TAG>_1a flags.  If you also constrain your answer between the
% START_CODE_HERE and END_CODE_HERE flags, your LaTeX will be styled as a
% solution within the final document.

% Please do not use the '<SCPD_SUBMISSION_TAG>' character anywhere within your code.  As expected,
% that will confuse the regular expressions we use to identify your solution.
\def\assignmentnum{2 }
\def\assignmenttitle{XCS229i Problem Set \assignmentnum}
\newcommand{\newsec}{\section}
\newcommand{\denselist}{\itemsep 0pt\partopsep 0pt}
\newcommand{\bitem}{\begin{itemize}\denselist}
\newcommand{\eitem}{\end{itemize}}
\newcommand{\benum}{\begin{enumerate}\denselist}
\newcommand{\eenum}{\end{enumerate}}

\newcommand{\fig}[1]{\private{\begin{center}
{\Large\bf ({#1})}
\end{center}}}

\newcommand{\cpsf}[1]{{\centerline{\psfig{#1}}}}
\newcommand{\mytitle}[1]{\centerline{\LARGE\bf #1}}

\newcommand{\myw}{{\bf w}}

\newcommand{\mypar}[1]{\vspace{1ex}\noindent{\bf {#1}}}

\def\thmcolon{\hspace{-.85em} {\bf :} }

\newtheorem{THEOREM}{Theorem}[section]
\newenvironment{theorem}{\begin{THEOREM} \thmcolon }%
                        {\end{THEOREM}}
\newtheorem{LEMMA}[THEOREM]{Lemma}
\newenvironment{lemma}{\begin{LEMMA} \thmcolon }%
                      {\end{LEMMA}}
\newtheorem{COROLLARY}[THEOREM]{Corollary}
\newenvironment{corollary}{\begin{COROLLARY} \thmcolon }%
                          {\end{COROLLARY}}
\newtheorem{PROPOSITION}[THEOREM]{Proposition}
\newenvironment{proposition}{\begin{PROPOSITION} \thmcolon }%
                            {\end{PROPOSITION}}
\newtheorem{DEFINITION}[THEOREM]{Definition}
\newenvironment{definition}{\begin{DEFINITION} \thmcolon \rm}%
                            {\end{DEFINITION}}
\newtheorem{CLAIM}[THEOREM]{Claim}
\newenvironment{claim}{\begin{CLAIM} \thmcolon \rm}%
                            {\end{CLAIM}}
\newtheorem{EXAMPLE}[THEOREM]{Example}
\newenvironment{example}{\begin{EXAMPLE} \thmcolon \rm}%
                            {\end{EXAMPLE}}
\newtheorem{REMARK}[THEOREM]{Remark}
\newenvironment{remark}{\begin{REMARK} \thmcolon \rm}%
                            {\end{REMARK}}
%\newenvironment{proof}{\noindent {\bf Proof:} \hspace{.677em}}%
%                      {}

%theorem
\newcommand{\thm}{\begin{theorem}}
%lemma
\newcommand{\lem}{\begin{lemma}}
%proposition
\newcommand{\pro}{\begin{proposition}}
%definition
\newcommand{\dfn}{\begin{definition}}
%remark
\newcommand{\rem}{\begin{remark}}
%example
\newcommand{\xam}{\begin{example}}
%corollary
\newcommand{\cor}{\begin{corollary}}
%proof
\newcommand{\prf}{\noindent{\bf Proof:} }
%end theorem
\newcommand{\ethm}{\end{theorem}}
%end lemma
\newcommand{\elem}{\end{lemma}}
%end proposition
\newcommand{\epro}{\end{proposition}}
%end definition
\newcommand{\edfn}{\bbox\end{definition}}
%end remark
\newcommand{\erem}{\bbox\end{remark}}
%end example
\newcommand{\exam}{\bbox\end{example}}
%end corollary
\newcommand{\ecor}{\end{corollary}}
%end proof
\newcommand{\eprf}{\bbox\vspace{0.1in}}
%begin equation
\newcommand{\beqn}{\begin{equation}}
%end equation
\newcommand{\eeqn}{\end{equation}}

%\newcommand{\eqref}[1]{Eq.~\ref{#1}}

\newcommand{\KB}{\mbox{\it KB\/}}
\newcommand{\infers}{\vdash}
\newcommand{\sat}{\models}
\newcommand{\bbox}{\vrule height7pt width4pt depth1pt}

\newcommand{\act}[1]{\stackrel{{#1}}{\rightarrow}}
\newcommand{\at}[1]{^{(#1)}}

\newcommand{\argmax}{{\rm argmax}}

\newcommand{\rimp}{\Rightarrow}
\newcommand{\dimp}{\Leftrightarrow}

\newcommand{\bX}{\mbox{\boldmath $X$}}
\newcommand{\bY}{\mbox{\boldmath $Y$}}
\newcommand{\bZ}{\mbox{\boldmath $Z$}}
\newcommand{\bU}{\mbox{\boldmath $U$}}
\newcommand{\bE}{\mbox{\boldmath $E$}}
\newcommand{\bx}{\mbox{\boldmath $x$}}
\newcommand{\be}{\mbox{\boldmath $e$}}
\newcommand{\by}{\mbox{\boldmath $y$}}
\newcommand{\bz}{\mbox{\boldmath $z$}}
\newcommand{\bu}{\mbox{\boldmath $u$}}
\newcommand{\bd}{\mbox{\boldmath $d$}}
\newcommand{\smbx}{\mbox{\boldmath $\scriptstyle x$}}
\newcommand{\smbd}{\mbox{\boldmath $\scriptstyle d$}}
\newcommand{\smby}{\mbox{\boldmath $\scriptstyle y$}}
\newcommand{\smbe}{\mbox{\boldmath $\scriptstyle e$}}

\newcommand{\Parents}{\mbox{\it Parents\/}}
\newcommand{\B}{{\cal B}}
\newcommand{\calH}{{\cal H}}

\newcommand{\word}[1]{\mbox{\it #1\/}}
\newcommand{\Action}{\word{Action}}
\newcommand{\Proposition}{\word{Proposition}}
\newcommand{\true}{\word{true}}
\newcommand{\false}{\word{false}}
\newcommand{\Pre}{\word{Pre}}
\newcommand{\Add}{\word{Add}}
\newcommand{\Del}{\word{Del}}
\newcommand{\Result}{\word{Result}}
\newcommand{\Regress}{\word{Regress}}
\newcommand{\Maintain}{\word{Maintain}}

\newcommand{\bor}{\bigvee}
\newcommand{\invert}[1]{{#1}^{-1}}

\newcommand{\commentout}[1]{}

\newcommand{\bmu}{\mbox{\boldmath $\mu$}}
\newcommand{\btheta}{\mbox{\boldmath $\theta$}}
\newcommand{\IR}{\mbox{$I\!\!R$}}

\newcommand{\tval}[1]{{#1}^{1}}
\newcommand{\fval}[1]{{#1}^{0}}

\newcommand{\tr}{{\rm tr}}
\newcommand{\vecy}{{\vec{y}}}
\renewcommand{\Re}{{\mathbb R}}

\def\twofigbox#1#2{%
\noindent\begin{minipage}{\textwidth}%
\epsfxsize=0.35\maxfigwidth
\noindent \epsffile{#1}\hfill
\epsfxsize=0.35\maxfigwidth
\epsffile{#2}\\
\makebox[0.35\textwidth]{(a)}\hfill\makebox[0.35\textwidth]{(b)}%
\end{minipage}}

\def\twofigboxcd#1#2{%
\noindent\begin{minipage}{\textwidth}%
\epsfxsize=0.35\maxfigwidth
\noindent \epsffile{#1}\hfill
\epsfxsize=0.35\maxfigwidth
\epsffile{#2}\\
\makebox[0.35\textwidth]{(c)}\hfill\makebox[0.35\textwidth]{(d)}%
\end{minipage}}

\def\twofigboxnolabel#1#2{%
\begin{minipage}{\textwidth}%
\epsfxsize=0.35\maxfigwidth
\noindent \epsffile{#1}\hfill
\epsfxsize=0.35\maxfigwidth
\epsffile{#2}\\
%\makebox[0.48\textwidth]{(a)}\hfill\makebox[0.48\textwidth]{(b)}%
\end{minipage}
}

\def\twofigboxnolabelFive#1#2{%
\begin{minipage}{\textwidth}%
\hbox to 0.5in{}\epsfxsize=0.35\maxfigwidth
\noindent \epsffile{#1}\hfill
\epsfxsize=0.35\maxfigwidth
\epsffile{#2}\hbox to 0.5in{}\\
%\makebox[0.48\textwidth]{(a)}\hfill\makebox[0.48\textwidth]{(b)}%
\end{minipage}
}

\def\threefigbox#1#2#3{%
\noindent\begin{minipage}{\textwidth}%
\epsfxsize=0.33\maxfigwidth
\noindent \epsffile{#1}\hfill
\epsfxsize=0.33\maxfigwidth
\noindent \epsffile{#2}\hfill 
\epsfxsize=0.33\maxfigwidth
\epsffile{#3}\\
\makebox[0.31\textwidth]{{\scriptsize (a)}}\hfill%
\makebox[0.31\textwidth]{{\scriptsize (b)}}\hfill
\makebox[0.31\textwidth]{{\scriptsize (c)}}%
\smallskip
\end{minipage}}

\def\threefigboxnolabel#1#2#3{%
\noindent\begin{minipage}{\textwidth}%
\epsfxsize=0.33\maxfigwidth
\noindent \epsffile{#1}\hfill
\epsfxsize=0.33\maxfigwidth
\noindent \epsffile{#2}\hfill 
\epsfxsize=0.33\maxfigwidth
\epsffile{#3}\\
%\makebox[0.31\textwidth]{{\scriptsize (a)}}\hfill%
%\makebox[0.31\textwidth]{{\scriptsize (b)}}\hfill
%\makebox[0.31\textwidth]{{\scriptsize (c)}}%
%\smallskip
\end{minipage}}

\newlength{\maxfigwidth}
\setlength{\maxfigwidth}{\textwidth}
%\def\captionsize {\footnotesize}
\def\captionsize {}

\newcommand{\xsi}{{x^{(i)}}}
\newcommand{\xsd}{{x^{(d)}}}
\newcommand{\xsvcd}{{x^{(\vcd)}}}
\newcommand{\xsj}{{x^{(j)}}}
\newcommand{\ysi}{{y^{(i)}}}
\newcommand{\ysj}{{y^{(j)}}}
\newcommand{\gsi}{{\gamma^{(i)}}}
\newcommand{\wsi}{{w^{(i)}}}
\newcommand{\esi}{{\epsilon^{(i)}}}
\newcommand{\calN}{{\cal N}}
\newcommand{\calX}{{\cal X}}
\newcommand{\calY}{{\cal Y}}
\newcommand{\calL}{{\cal L}}
\newcommand{\calP}{{\cal P}}
\newcommand{\calD}{{\cal D}}
\newcommand{\ytil}{{\tilde{y}}}

\newcommand{\Ber}{{\rm Bernoulli}}
\newcommand{\E}{{\rm E}}

\newcommand{\pstar}{{p^{\ast}}}
\newcommand{\bstar}{{b^{\ast}}}
\newcommand{\dstar}{{d^{\ast}}}
\newcommand{\wstar}{{w^{\ast}}}
\newcommand{\alphastar}{\alpha^{\ast}}
\newcommand{\alphastari}{{\alpha_i^{\ast}}}
\newcommand{\betastar}{{\beta^{\ast}}}
\newcommand{\tol}{{\textit tol}}
\newcommand{\phihat}{\hat\phi}
\newcommand{\ehat}{\hat\varepsilon}
\newcommand{\hhat}{\hat{h}}
\newcommand{\hstar}{h^\ast}
\newcommand{\VC}{{\rm VC}}

\newcommand{\hwb}{{h_{w,b}}}

\begin{document}
\pagestyle{myheadings} \markboth{}{\assignmenttitle}

% <SCPD_SUBMISSION_TAG>_entire_submission

This handout includes space for every question that requires a written response.
Please feel free to use it to handwrite your solutions (legibly, please).  If
you choose to typeset your solutions, the |README.md| for this assignment includes
instructions to regenerate this handout with your typeset \LaTeX{} solutions.
\ruleskip

\LARGE
1.a
\normalsize

% <SCPD_SUBMISSION_TAG>_1a
\begin{answer}
  Since $g'(z) = g(z)(1-g(z))$ and $h(x) = g(\theta^T x)$, it follows that $\partial h(x) / \partial \theta_k = h(x)(1 - h(x)) x_k$.

  Letting $h_{\theta}(x^{(i)}) = g(\theta^T x^{(i)})
  = 1/(1 + \exp(-\theta^T x^{(i)}))$, we have\\

  \begin{flalign*}
    \frac{\partial\log h_{\theta}(x^{(i)})}{\partial\theta_k} &= \\
    % ### START CODE HERE ###
    \frac{1}{h_{\theta}(x^{(i)})} \frac{\partial h_{\theta}(x^{(i)})}{\partial\theta_k} &= \\
     \frac{1}{h_{\theta}(x^{(i)})} h_{\theta}(x^{(i)})(1-h_{\theta}(x^{(i)}))x_k &= \\
     (1-h_{\theta}(x^{(i)}))x_k &= \\
     \\
    % ### END CODE HERE ###
    \frac{\partial\log(1 - h_{\theta}(x^{(i)}))}{\partial\theta_k} &= \\
    % ### START CODE HERE ###
        \frac{1}{1 - h_{\theta}(x^{(i)})} \frac{\partial (1 - h_{\theta}(x^{(i)}))}{\partial\theta_k} &= \\
     (-h_{\theta}(x^{(i)}))x_k &= \\
     \\
    % ### END CODE HERE ###
  \end{flalign*}

  Substituting into our equation for $J(\theta)$, we have
  %
  \begin{flalign*}
    \frac{\partial J(\theta)}{\partial\theta_k} &=\\
    % ### START CODE HERE ###
    -\frac{1}{\nexp}\sum_{i=1}^{\nexp} ( y^{(i)} x_k(1-h_{\theta}(x^{(i)})) + (1-y^{(i)}) (-h_{\theta}(x^{(i)}))x^{(i)}_k &= \\
    -\frac{1}{\nexp}\sum_{i=1}^{\nexp} ( y^{(i)} -h_{\theta}(x^{(i)}))x^{(i)}_k
    % ### END CODE HERE ###
  \end{flalign*}
  
  Consequently, the $(k, l)$ entry of the Hessian is given by
  
  \begin{flalign*}
    H_{kl} = \frac{\partial^2 J(\theta)}{\partial\theta_k\partial\theta_l} &=\\
    % ### START CODE HERE ###
    -\frac{1}{\nexp}\sum_{i=1}^{\nexp} \frac{\partial}{\partial\theta_l}( y^{(i)} -h_{\theta}(x^{(i)}))x^{(i)}_k &= \\
    \frac{1}{\nexp}\sum_{i=1}^{\nexp}  h_{\theta}(x^{(i)}) (1 -h_{\theta}(x^{(i)})) x^{(i)}_k x^{(i)}_l) &= \\
    % ### END CODE HERE ###
  \end{flalign*}
  
  Using the fact that $X_{ij} = x_i x_j$ if and only if $X = xx^T$, we have
  
  \begin{flalign*}
    H &= \\
    % ### START CODE HERE ###
        \frac{1}{\nexp}\sum_{i=1}^{\nexp}  h(x^{(i)}) (1 -h(x^{(i)}))) x^{(i)} x^{(i)T} &= \\
    % ### END CODE HERE ###
  \end{flalign*}

  To prove that $H$ is positive semi-definite, show $z^T Hz \ge 0$ for all $z\in\Re^\di$.
  
  \begin{flalign*}
    z^T H z &=\\
    % ### START CODE HERE ###
    z^T (\frac{1}{\nexp}\sum_{i=1}^{\nexp}  h(x^{(i)}) (1 -h(x^{(i)})))x^{(i)} x^{(i)T} ) z &= \\
    \frac{1}{\nexp}\sum_{i=1}^{\nexp}  h(x^{(i)}) (1 -h(x^{(i)}))z^Tx^{(i)} x^{(i)T} z &= \\
    \frac{1}{\nexp}\sum_{i=1}^{\nexp} h(x^{(i)}) (1 -h(x^{(i)})) (x^T z)^2  &= \\
    % ### END CODE HERE ###
  \end{flalign*}
  
  % ### START CODE HERE ###\texttt{\texttt{•}}
 Now we know that \\
  $ h(x^{(i)}) (1 -h(x^{(i)})) \in [1, 0] $ \\
  Also  more generally it is clear that \\
  $ (x^Tz )^2 \geqslant 0 $ \\ 
  Hence for any vector z, it holds true that
  $z^THz \geqslant 0 $
  % ### END CODE HERE ###
\end{answer}
% <SCPD_SUBMISSION_TAG>_1a
\clearpage

\LARGE
1.c
\normalsize

% <SCPD_SUBMISSION_TAG>_1c
\begin{answer}
  For shorthand, we let $\mc{H} = \{\phi, \Sigma, \mu_{0}, \mu_1\}$ denote
  the parameters for the problem.
  Since the given formulae are conditioned on $y$, use Bayes rule to get:
  \begin{align*}
    p(y =1\vert  x ; \mc{H}) &= \frac {p(x\vert y=1; \mc{H}) p(y=1; \mc{H})} {p(x; \mc{H})}\\
    & = \frac {p(x\vert y=1; \mc{H}) p(y=1; \mc{H})}
      {p(x\vert y=1; \mc{H}) p(y=1; \mc{H}) + p(x\vert y={0}; \mc{H}) p(y={0};
      \mc{H})}\\
    &\\
    % ### START CODE HERE ###
    &= \frac { \frac{1}{(2\pi)^{\di/2} \Sigma^{1/2}}\exp\left(-\frac{1}{2} (x-\mu_1)^T \Sigma^{-1} (x-\mu_1) \right) (\phi) }
      {\frac{1}{(2\pi)^{\di/2} \Sigma^{1/2}}\exp\left(-\frac{1}{2} (x-\mu_1)^T \Sigma^{-1} (x-\mu_1) \right) (\phi)  + \frac{1}{(2\pi)^{\di/2} \Sigma^{1/2}}\exp\left(-\frac{1}{2} (x-\mu_0)^T \Sigma^{-1} (x-\mu_0) \right) (1- \phi) }\\
    &\\ 
    &= \frac { (\phi) \exp\left(-\frac{1}{2} (x-\mu_1)^T \Sigma^{-1} (x-\mu_1) \right)  }
      { (\phi) \exp\left(-\frac{1}{2} (x-\mu_1)^T \Sigma^{-1} (x-\mu_1) \right) + (1- \phi)exp\left(-\frac{1}{2} (x-\mu_0)^T \Sigma^{-1} (x-\mu_0) \right)  }\\
    &\\ 
    &= \frac { 1 }
      { 1 + \frac{(1- \phi)exp\left(-\frac{1}{2} (x-\mu_0)^T \Sigma^{-1} (x-\mu_0) \right)}{(\phi) \exp\left(-\frac{1}{2} (x-\mu_1)^T \Sigma^{-1} (x-\mu_1) \right) } }\\
    &\\ 
    &= \frac { 1 }
      { 1 + \exp( \log(\frac{1-\phi}{\phi}) \left(-\frac{1}{2} (x-\mu_0)^T \Sigma^{-1} (x-\mu_0) \right) + \left(-\frac{1}{2} (x-\mu_1)^T \Sigma^{-1} (x-\mu_1) \right)  )}\\
    &\\
    &= \frac { 1 }
      { 1 + \exp( \log(\frac{1-\phi}{\phi}) -\frac{1}{2} (x^T \Sigma^{-1} x -2 x^T\Sigma^{-1}\mu_0+ \mu_0^T \Sigma^{-1} \mu_0) + \frac{1}{2} (x^T \Sigma^{-1} x -2 x^T\Sigma^{-1}\mu_1+ \mu_1^T \Sigma^{-1} \mu_1       )}\\
    &\\ 
&= \frac { 1 }
      { 1 + \exp( \log(\frac{1-\phi}{\phi}) + x^T \Sigma^{-1} \mu_0 - x^T \Sigma^{-1} \mu_1 -\frac{1}{2} \mu_0^T \Sigma^{-1} \mu_0  + \frac{1}{2} \mu_1^T \Sigma^{-1} \mu_1}\\
    &\\ 
    &= \frac { 1 }
      { 1 + \exp( \log(\frac{1-\phi}{\phi}) + x^T \Sigma^{-1} (\mu_0 -  \mu_1) -\frac{1}{2} \mu_0^T \Sigma^{-1} \mu_0  + \frac{1}{2} \mu_1^T \Sigma^{-1} \mu_1}\\
    &\\ 
    & Let \theta = \Sigma^{-1} (\mu_1 -  \mu_0) &\\ 
    & Let \theta_0 =  \frac{1}{2}(\mu_0^T \Sigma^{-1}  \mu_0  - \mu_1^T  \Sigma^{-1} \mu_1) + \log(\frac{1-\phi}{\phi})&\\ 
    & p(y =1\vert  x ; \mc{H}) = \frac { 1 }    { 1 + \exp(-(\theta^Tx + \theta_0))}
    &\\ 
    % ### END CODE HERE ###
  \end{align*}
\end{answer}
% <SCPD_SUBMISSION_TAG>_1c
\clearpage

\LARGE
1.d
\normalsize

% <SCPD_SUBMISSION_TAG>_1d
\begin{answer}
  First, derive the expression for the log-likelihood of the training data:
  \begin{flalign*}
    \ell(\phi, \mu_{0}, \mu_1, \Sigma) &= \log \prod_{i=1}^\nexp p(x^{(i)} \vert  y^{(i)}; \mu_{0}, \mu_1, \Sigma) p(y^{(i)}; \phi)\\
    &= \sum_{i=1}^{\nexp} \log p(x^{(i)} \vert  y^{(i)}; \mu_{0}, \mu_1, \Sigma) +
    \sum_{i=1}^{n} \log p(y^{(i)}; \phi)\\
    &\\
    % ### START CODE HERE ###
    &= \sum_{i=1}^{\nexp} [\log \frac{1}{(2\pi)^{\di/2} \vert \Sigma\vert^{1/2}}-\frac{1}{2} (x^{(i)}-\mu_{y^{(i)}} )^T \Sigma^{-1} (x^{(i)}-\mu_{y^{(i)}} ) +
    \log\phi^{y^{(i)}} +\log(1-\phi)^{(1- y^{(i)})}]\\
    &\\
    &= \sum_{i=1}^{\nexp}  [-\frac{n}{2}\log(2\pi) -\frac{1}{2}\log\vert \Sigma\vert -\frac{1}{2} (x^{(i)}-\mu_{y^{(i)}} )^T \Sigma^{-1} (x^{(i)}-\mu_{y^{(i)}} ) +y^{(i)}\log\phi + (1- y^{(i)})\log(1-\phi)]
     \\
    &\\
    \textrm{removing constant terms}
    &= \sum_{i=1}^{\nexp} [-\frac{1}{2}\log\vert \Sigma\vert -\frac{1}{2} (x^{(i)}-\mu_{y^{(i)}} )^T \Sigma^{-1} (x^{(i)}-\mu_{y^{(i)}} ) +y^{(i)}\log\phi + (1- y^{(i)})\log(1-\phi)]
     \\
    &\\
    % ### END CODE HERE ###
  \end{flalign*}

  Now, the likelihood is maximized by setting the derivative (or gradient) with respect to each of the parameters to zero.\\

  \textbf{For $\mathbf{\phi}$:}

  \begin{flalign*}
    \frac{\partial \ell}{\partial \phi}&=\\
    % ### START CODE HERE ###
    &0 = \sum_{i=1}^{\nexp} (\frac{y^{(i)}}{\phi} - \frac{(1- y^{(i)}}{(1-\phi)})    \\
    &\\
    & \sum_{i=1}^{\nexp} (\frac{y^{(i)}}{\phi})  = \sum_{i=1}^{\nexp} \frac{(1- y^{(i)}}{(1-\phi)})    \\
    &\\
    & \sum_{i=1}^{\nexp} y^{(i)} (1-\phi) = \sum_{i=1}^{\nexp} \phi (1- y^{(i)})    \\
    &\\
    & \phi = \frac{1}{n}\sum_{i=1}^{\nexp} 1 \{y^{(i)} = 1\}   \\
    &\\
    % ### END CODE HERE ###
  \end{flalign*}

  Setting this equal to zero and solving for $\phi$ gives the maximum
  likelihood estimate.\\

  \textbf{For $\mathbf{\mu_0}$:}

  {\bf Hint:}  Remember that $\Sigma$ (and thus $\Sigma^{-1}$) is symmetric.

  \begin{flalign*}
    \nabla_{\mu_{0}}\ell &=\\
    % ### START CODE HERE ###
    &= -\frac{1}{2}\sum_{y^{(i)}=0}^{} \nabla_{\mu_{0}} [(x^{(i)}-\mu_{0} )^T \Sigma^{-1} (x^{(i)}-\mu_{0} )]
     \\
    &\\
    &= -\frac{1}{2}\sum_{y^{(i)}=0}^{} \nabla_{\mu_{0}} [(x^{(i)T}\Sigma^{-1}x^{(i)} + \mu_{0}^T\Sigma^{-1}\mu_{0}  -2 \mu_{0}^T\Sigma^{-1}x^{(i)}]
     \\
    &\\
    &= -\frac{1}{2}\sum_{y^{(i)}=0}^{} \nabla_{\mu_{0}} [ \mu_{0}^T\Sigma^{-1}\mu_{0}  -2 \mu_{0}^T\Sigma^{-1}x^{(i)}]
     \\
    &\\
    &= -\frac{1}{2}\sum_{y^{(i)}=0}^{} [2 \Sigma^{-1}\mu_{0}  -2 \Sigma^{-1}x^{(i)}] = 0
     \\
    &\\
    &\\
    &0 = \sum_{y^{(i)}=0}^{} [\Sigma^{-1}x^{(i)} - \Sigma^{-1}\mu_{0} ]
     \\
    &\\
    & \Sigma^{-1}\mu_{0} \sum_{i=1}^{n}1\{ y^{(i)} = 0\} = \Sigma^{-1}\sum_{i=1}^{n}1\{ y^{(i)} = 0\} x^{(i)} 
     \\
    &\\
    &\\
    & \mu_{0} = \frac{\sum_{i=1}^{n}1\{ y^{(i)} = 0\} x^{(i)} }{\sum_{i=1}^{n}1\{ y^{(i)} = 0\}}
     \\
    &\\
    % ### END CODE HERE ###
  \end{flalign*}

  Setting this gradient to zero gives the maximum likelihood estimate
  for $\mu_{0}$.\\

  \textbf{For $\mathbf{\mu_1}$:}

  {\bf Hint:}  Remember that $\Sigma$ (and thus $\Sigma^{-1}$) is symmetric.

  \begin{flalign*}
    \nabla_{\mu_{1}}\ell &=\\
    % ### START CODE HERE ###
        &= -\frac{1}{2}\sum_{y^{(i)}=1}^{} \nabla_{\mu_{1}} [(x^{(i)}-\mu_{1} )^T \Sigma^{-1} (x^{(i)}-\mu_{1} )]
     \\
    &\\
    &= -\frac{1}{2}\sum_{y^{(i)}=1}^{} \nabla_{\mu_{1}} [(x^{(i)T}\Sigma^{-1}x^{(i)} + \mu_{1}^T\Sigma^{-1}\mu_{1}  -2 \mu_{1}^T\Sigma^{-1}x^{(i)}]
     \\
    &\\
    &= -\frac{1}{2}\sum_{y^{(i)}=1}^{} \nabla_{\mu_{1}} [ \mu_{1}^T\Sigma^{-1}\mu_{1}  -2 \mu_{1}^T\Sigma^{-1}x^{(i)}]
     \\
    &\\
    &= -\frac{1}{2}\sum_{y^{(i)}=1}^{} [2 \Sigma^{-1}\mu_{1}  -2 \Sigma^{-1}x^{(i)}] = 0
     \\
    &\\
    &\\
    &0 = \sum_{y^{(i)}=1}^{} [\Sigma^{-1}x^{(i)} - \Sigma^{-1}\mu_{1} ]
     \\
    &\\
    & \Sigma^{-1}\mu_{1} \sum_{i=1}^{n}1\{ y^{(i)} = 0\} = \Sigma^{-1}\sum_{i=1}^{n}1\{ y^{(i)} = 0\} x^{(i)} 
     \\
    &\\
    &\\
    & \mu_{1} = \frac{\sum_{i=1}^{n}1\{ y^{(i)} = 1\} x^{(i)} }{\sum_{i=1}^{n}1\{ y^{(i)} = 1\}}
     \\
    % ### END CODE HERE ###
  \end{flalign*}

  Setting this gradient to zero gives the maximum likelihood estimate
  for $\mu_{1}$.\\

  For $\Sigma$, we find the gradient with respect to $S = \Sigma^{-1}$ rather than $\Sigma$ just to simplify the derivation (note that $\vert S\vert  = \frac{1}{\vert \Sigma\vert }$).
  You should convince yourself that the maximum likelihood estimate $S_\nexp$ found in this way would correspond to the actual maximum likelihood estimate $\Sigma_\nexp$ as $S_\nexp^{-1} = \Sigma_\nexp$.

  {\bf Hint:}  You may need the following identities: 
  \begin{equation*}
    \nabla_S \vert S\vert  = \vert S\vert  (S^{-1})^T
  \end{equation*}
  \begin{equation*}
    \nabla_S b_i^T S b_i = \nabla_S tr \left( b_i^T S b_i \right) =
    \nabla_S tr \left( S b_i b_i^T \right) = b_i b_i^T
  \end{equation*}

  \begin{flalign*}
    \nabla_S\ell &=\\
    % ### START CODE HERE ###
     &= -\frac{1}{2} \sum_{i=1}^{\nexp} \nabla_S [-\log\vert S\vert + (x^{(i)}-\mu_{y^{(i)}} )^T S (x^{(i)}-\mu_{y^{(i)}} )]
     \\
    &\\
    &= -\frac{1}{2} \sum_{i=1}^{\nexp} [- S^{-1} + (x^{(i)}-\mu_{y^{(i)}} ) (x^{(i)}-\mu_{y^{(i)}} )^T ]
     \\
    &\\
    &=  \sum_{i=1}^{\nexp} \frac{1}{2} \Sigma -\frac{1}{2} \sum_{i=1}^{\nexp} [(x^{(i)}-\mu_{y^{(i)}} ) (x^{(i)}-\mu_{y^{(i)}} )^T]
     \\
    &\\
    &0 =  \sum_{i=1}^{\nexp} \frac{1}{2} \Sigma -\frac{1}{2} \sum_{i=1}^{\nexp} [(x^{(i)}-\mu_{y^{(i)}} ) (x^{(i)}-\mu_{y^{(i)}} )^T]
     \\
    &\\
    &\Sigma  =  \frac{1}{n}\sum_{i=1}^{\nexp} (x^{(i)}-\mu_{y^{(i)}} ) (x^{(i)}-\mu_{y^{(i)}} )^T 
     \\
    &\\
  % ### END CODE HERE ###
  \end{flalign*}

  Next, substitute $\Sigma = S^{-1}$.  Setting this gradient to zero gives the required maximum likelihood estimate for $\Sigma$.\\
\end{answer}
% <SCPD_SUBMISSION_TAG>_1d
\clearpage

\LARGE
1.f
\normalsize

% <SCPD_SUBMISSION_TAG>_1f
\begin{answer}
  % ### START CODE HERE ###
  For Dataset 1 \\
  LR Accuracy obtained was 0.83 \\
  GDA Accuracy obtained was 0.81 \\
  It can be seen that Logistic regression performed better than GDA for Dataset 1 \\ 
  Also looking at the shape of the data points we can see that it doesn’t follow Gaussian distribution and it would perform worse as demonstrated above. \\
  
  % ### END CODE HERE ###
\end{answer}
% <SCPD_SUBMISSION_TAG>_1f
\clearpage

\LARGE
1.g
\normalsize

% <SCPD_SUBMISSION_TAG>_1g
\begin{answer}
  % ### START CODE HERE ###
  (From pervious answer 1.h) \\
  For Dataset 1 \\
  LR Accuracy obtained was 0.83 \\
  GDA Accuracy obtained was 0.81 \\
  It can be seen that Logistic regression performed better than GDA for Dataset 1 \\ 
  Also looking at the shape of the data points we can see that it doesn’t follow Gaussian distribution and it would perform worse as demonstrated above. \\

(After running test case 1g-0-basic) \\
For Dataset 2 \\
  LR Accuracy obtained was 0.86 \\
  GDA Accuracy obtained was 0.86 \\
  It can be seen clearly that GDA performed worst than Logistic regression on Dataset1 (as against to Dataset2)\\
  Looking at the shape of data points in Dataset2 it appears that data is more normally distributed and hence GDA did not perform worst then LR.
  
  % ### END CODE HERE ###
\end{answer}
% <SCPD_SUBMISSION_TAG>_1g
\clearpage

\LARGE
1.h
\normalsize

% <SCPD_SUBMISSION_TAG>_1h
\begin{answer}
  % ### START CODE HERE ###
  There can be various transformation that can be applied on Dataset1 so that GDA will perform better. \\
  Some sample transformation to make data more gaussian like \\
  1. Log transformation \\
  2. Square root transformation\\
  3. Box-Cox transformation \\
  4. Yeo-Johnson transformation \\
  Since data set contains negative values Yeo Johnson transformation will improve performance of GDA.
  % ### END CODE HERE ###
\end{answer}
% <SCPD_SUBMISSION_TAG>_1h
\clearpage

\LARGE
2.a
\normalsize

% <SCPD_SUBMISSION_TAG>_2a
\begin{answer}
  % ### START CODE HERE ###
  Lets start with poisson distribution parameterized by $\lambda$ \\
  $ p(y; \lambda) = \frac{e^{-\lambda}\lambda^y}{y!} $ \\
  $ p(y; \lambda) = \exp(\log(\frac{e^{-\lambda}\lambda^y}{y!})) $ \\
  $ p(y; \lambda) = \frac{1}{y!} \exp(y \log\lambda -\lambda )$ \\ \\
  This can be written in exponential family as $ b(y) \exp(\eta^TT(y)-a(\eta)) $ \\ \\
  So values of various terms will be \\ \\  $b(y) = \frac{1}{y!} $ \\
  $T(y) = y $ \\
  $ \eta = \log\lambda$ \\
  $ a(\eta) = e^\eta =\lambda$ \\
  % ### END CODE HERE ###
\end{answer}
% <SCPD_SUBMISSION_TAG>_2a
\clearpage

\LARGE
2.b
\normalsize

% <SCPD_SUBMISSION_TAG>_2b
\begin{answer}
  % ### START CODE HERE ###
 $  g(\eta) = E[y;\eta] $ \\ 
 $  g(\eta) = \lambda $ \\ 
 $  g(\eta) = e^\eta $ \\ 
  % ### END CODE HERE ###
\end{answer}
% <SCPD_SUBMISSION_TAG>_2b
\clearpage

\LARGE
2.c
\normalsize

% <SCPD_SUBMISSION_TAG>_2c
\begin{answer}
  The log-likelihood of an example $(x^{(i)}, y^{(i)})$ is defined as
  $\ell(\theta) = \log p(y^{(i)} \vert  x^{(i)}; \theta)$. To derive the stochastic
  gradient ascent rule, use the results in part (a) and the standard GLM
  assumption that $\eta = \theta^Tx$.
  \begin{flalign*}
    \frac{\partial \ell(\theta)}{\partial \theta_j}
    &= \frac{\partial \log p(y^{(i)} \vert  x^{(i)}; \theta)}{\partial \theta_j}\\
    &= \frac {\partial \log \left({\frac{1}{y^{(i)}!} \exp(\eta^T y^{(i)} -
    e^\eta)}\right)} {\partial \theta_j}\\
    &\\
    % ### START CODE HERE ###
    &= \frac {\partial \log \left({\frac{1}{y^{(i)}!} \exp((\theta^Tx)^T y^{(i)} -
    e^{\theta^Tx})}\right)} {\partial \theta_j}\\
    &\\
    &= \frac {\partial \log \left({\frac{1}{y^{(i)}!} \exp(x^{(i)T} \theta y^{(i)} -
    e^{x^{(i)T\theta}})}\right)} {\partial \theta_j}\\
    &\\
    &= \frac {\partial (- \log ({y^{(i)}!) +x^{(i)T} \theta y^{(i)} -
    e^{x^{(i)T\theta}})} } {\partial \theta_j}\\
    &\\
    &=  x^{(i)T}y^{(i)} -  x^{(i)T}e^{x^{(i)T\theta}} \\
    &\\
    &\\
    &=  (y^{(i)} -  e^{\theta^Tx^{(i)}})x^{(i)}_j \\
    &\\
    % ### END CODE HERE ###
  \end{flalign*}

  Thus the stochastic gradient ascent update rule should be:

  \begin{equation*}
  \theta_j := \theta_j + \alpha \frac{\partial \ell(\theta)}{\partial \theta_j},
  \end{equation*}

  which reduces here to:
  % ### START CODE HERE ###
  $\theta_j := \theta_j + \alpha (y^{(i)} -  e^{\theta^Tx^{(i)}})x^{(i)}_j $
  % ### END CODE HERE ###
\end{answer}
% <SCPD_SUBMISSION_TAG>_2c
\clearpage

% <SCPD_SUBMISSION_TAG>_entire_submission

\end{document}