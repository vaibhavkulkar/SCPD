% This contents of this file will be inserted into the _Solutions version of the
% output tex document.  Here's an example:

% If assignment with subquestion (1.a) requires a written response, you will
% find the following flag within this document: <SCPD_SUBMISSION_TAG>_1a
% In this example, you would insert the LaTeX for your solution to (1.a) between
% the <SCPD_SUBMISSION_TAG>_1a flags.  If you also constrain your answer between the
% START_CODE_HERE and END_CODE_HERE flags, your LaTeX will be styled as a
% solution within the final document.

% Please do not use the '<SCPD_SUBMISSION_TAG>' character anywhere within your code.  As expected,
% that will confuse the regular expressions we use to identify your solution.
\def\assignmentnum{5 }
\def\assignmenttitle{XCS229i Problem Set \assignmentnum}
\newcommand{\newsec}{\section}
\newcommand{\denselist}{\itemsep 0pt\partopsep 0pt}
\newcommand{\bitem}{\begin{itemize}\denselist}
\newcommand{\eitem}{\end{itemize}}
\newcommand{\benum}{\begin{enumerate}\denselist}
\newcommand{\eenum}{\end{enumerate}}

\newcommand{\fig}[1]{\private{\begin{center}
{\Large\bf ({#1})}
\end{center}}}

\newcommand{\cpsf}[1]{{\centerline{\psfig{#1}}}}
\newcommand{\mytitle}[1]{\centerline{\LARGE\bf #1}}

\newcommand{\myw}{{\bf w}}

\newcommand{\mypar}[1]{\vspace{1ex}\noindent{\bf {#1}}}

\def\thmcolon{\hspace{-.85em} {\bf :} }

\newtheorem{THEOREM}{Theorem}[section]
\newenvironment{theorem}{\begin{THEOREM} \thmcolon }%
                        {\end{THEOREM}}
\newtheorem{LEMMA}[THEOREM]{Lemma}
\newenvironment{lemma}{\begin{LEMMA} \thmcolon }%
                      {\end{LEMMA}}
\newtheorem{COROLLARY}[THEOREM]{Corollary}
\newenvironment{corollary}{\begin{COROLLARY} \thmcolon }%
                          {\end{COROLLARY}}
\newtheorem{PROPOSITION}[THEOREM]{Proposition}
\newenvironment{proposition}{\begin{PROPOSITION} \thmcolon }%
                            {\end{PROPOSITION}}
\newtheorem{DEFINITION}[THEOREM]{Definition}
\newenvironment{definition}{\begin{DEFINITION} \thmcolon \rm}%
                            {\end{DEFINITION}}
\newtheorem{CLAIM}[THEOREM]{Claim}
\newenvironment{claim}{\begin{CLAIM} \thmcolon \rm}%
                            {\end{CLAIM}}
\newtheorem{EXAMPLE}[THEOREM]{Example}
\newenvironment{example}{\begin{EXAMPLE} \thmcolon \rm}%
                            {\end{EXAMPLE}}
\newtheorem{REMARK}[THEOREM]{Remark}
\newenvironment{remark}{\begin{REMARK} \thmcolon \rm}%
                            {\end{REMARK}}
%\newenvironment{proof}{\noindent {\bf Proof:} \hspace{.677em}}%
%                      {}

%theorem
\newcommand{\thm}{\begin{theorem}}
%lemma
\newcommand{\lem}{\begin{lemma}}
%proposition
\newcommand{\pro}{\begin{proposition}}
%definition
\newcommand{\dfn}{\begin{definition}}
%remark
\newcommand{\rem}{\begin{remark}}
%example
\newcommand{\xam}{\begin{example}}
%corollary
\newcommand{\cor}{\begin{corollary}}
%proof
\newcommand{\prf}{\noindent{\bf Proof:} }
%end theorem
\newcommand{\ethm}{\end{theorem}}
%end lemma
\newcommand{\elem}{\end{lemma}}
%end proposition
\newcommand{\epro}{\end{proposition}}
%end definition
\newcommand{\edfn}{\bbox\end{definition}}
%end remark
\newcommand{\erem}{\bbox\end{remark}}
%end example
\newcommand{\exam}{\bbox\end{example}}
%end corollary
\newcommand{\ecor}{\end{corollary}}
%end proof
\newcommand{\eprf}{\bbox\vspace{0.1in}}
%begin equation
\newcommand{\beqn}{\begin{equation}}
%end equation
\newcommand{\eeqn}{\end{equation}}

%\newcommand{\eqref}[1]{Eq.~\ref{#1}}

\newcommand{\KB}{\mbox{\it KB\/}}
\newcommand{\infers}{\vdash}
\newcommand{\sat}{\models}
\newcommand{\bbox}{\vrule height7pt width4pt depth1pt}

\newcommand{\act}[1]{\stackrel{{#1}}{\rightarrow}}
\newcommand{\at}[1]{^{(#1)}}

\newcommand{\argmax}{{\rm argmax}}

\newcommand{\rimp}{\Rightarrow}
\newcommand{\dimp}{\Leftrightarrow}

\newcommand{\bX}{\mbox{\boldmath $X$}}
\newcommand{\bY}{\mbox{\boldmath $Y$}}
\newcommand{\bZ}{\mbox{\boldmath $Z$}}
\newcommand{\bU}{\mbox{\boldmath $U$}}
\newcommand{\bE}{\mbox{\boldmath $E$}}
\newcommand{\bx}{\mbox{\boldmath $x$}}
\newcommand{\be}{\mbox{\boldmath $e$}}
\newcommand{\by}{\mbox{\boldmath $y$}}
\newcommand{\bz}{\mbox{\boldmath $z$}}
\newcommand{\bu}{\mbox{\boldmath $u$}}
\newcommand{\bd}{\mbox{\boldmath $d$}}
\newcommand{\smbx}{\mbox{\boldmath $\scriptstyle x$}}
\newcommand{\smbd}{\mbox{\boldmath $\scriptstyle d$}}
\newcommand{\smby}{\mbox{\boldmath $\scriptstyle y$}}
\newcommand{\smbe}{\mbox{\boldmath $\scriptstyle e$}}

\newcommand{\Parents}{\mbox{\it Parents\/}}
\newcommand{\B}{{\cal B}}
\newcommand{\calH}{{\cal H}}

\newcommand{\word}[1]{\mbox{\it #1\/}}
\newcommand{\Action}{\word{Action}}
\newcommand{\Proposition}{\word{Proposition}}
\newcommand{\true}{\word{true}}
\newcommand{\false}{\word{false}}
\newcommand{\Pre}{\word{Pre}}
\newcommand{\Add}{\word{Add}}
\newcommand{\Del}{\word{Del}}
\newcommand{\Result}{\word{Result}}
\newcommand{\Regress}{\word{Regress}}
\newcommand{\Maintain}{\word{Maintain}}

\newcommand{\bor}{\bigvee}
\newcommand{\invert}[1]{{#1}^{-1}}

\newcommand{\commentout}[1]{}

\newcommand{\bmu}{\mbox{\boldmath $\mu$}}
\newcommand{\btheta}{\mbox{\boldmath $\theta$}}
\newcommand{\IR}{\mbox{$I\!\!R$}}

\newcommand{\tval}[1]{{#1}^{1}}
\newcommand{\fval}[1]{{#1}^{0}}

\newcommand{\tr}{{\rm tr}}
\newcommand{\vecy}{{\vec{y}}}
\renewcommand{\Re}{{\mathbb R}}

\def\twofigbox#1#2{%
\noindent\begin{minipage}{\textwidth}%
\epsfxsize=0.35\maxfigwidth
\noindent \epsffile{#1}\hfill
\epsfxsize=0.35\maxfigwidth
\epsffile{#2}\\
\makebox[0.35\textwidth]{(a)}\hfill\makebox[0.35\textwidth]{(b)}%
\end{minipage}}

\def\twofigboxcd#1#2{%
\noindent\begin{minipage}{\textwidth}%
\epsfxsize=0.35\maxfigwidth
\noindent \epsffile{#1}\hfill
\epsfxsize=0.35\maxfigwidth
\epsffile{#2}\\
\makebox[0.35\textwidth]{(c)}\hfill\makebox[0.35\textwidth]{(d)}%
\end{minipage}}

\def\twofigboxnolabel#1#2{%
\begin{minipage}{\textwidth}%
\epsfxsize=0.35\maxfigwidth
\noindent \epsffile{#1}\hfill
\epsfxsize=0.35\maxfigwidth
\epsffile{#2}\\
%\makebox[0.48\textwidth]{(a)}\hfill\makebox[0.48\textwidth]{(b)}%
\end{minipage}
}

\def\twofigboxnolabelFive#1#2{%
\begin{minipage}{\textwidth}%
\hbox to 0.5in{}\epsfxsize=0.35\maxfigwidth
\noindent \epsffile{#1}\hfill
\epsfxsize=0.35\maxfigwidth
\epsffile{#2}\hbox to 0.5in{}\\
%\makebox[0.48\textwidth]{(a)}\hfill\makebox[0.48\textwidth]{(b)}%
\end{minipage}
}

\def\threefigbox#1#2#3{%
\noindent\begin{minipage}{\textwidth}%
\epsfxsize=0.33\maxfigwidth
\noindent \epsffile{#1}\hfill
\epsfxsize=0.33\maxfigwidth
\noindent \epsffile{#2}\hfill 
\epsfxsize=0.33\maxfigwidth
\epsffile{#3}\\
\makebox[0.31\textwidth]{{\scriptsize (a)}}\hfill%
\makebox[0.31\textwidth]{{\scriptsize (b)}}\hfill
\makebox[0.31\textwidth]{{\scriptsize (c)}}%
\smallskip
\end{minipage}}

\def\threefigboxnolabel#1#2#3{%
\noindent\begin{minipage}{\textwidth}%
\epsfxsize=0.33\maxfigwidth
\noindent \epsffile{#1}\hfill
\epsfxsize=0.33\maxfigwidth
\noindent \epsffile{#2}\hfill 
\epsfxsize=0.33\maxfigwidth
\epsffile{#3}\\
%\makebox[0.31\textwidth]{{\scriptsize (a)}}\hfill%
%\makebox[0.31\textwidth]{{\scriptsize (b)}}\hfill
%\makebox[0.31\textwidth]{{\scriptsize (c)}}%
%\smallskip
\end{minipage}}

\newlength{\maxfigwidth}
\setlength{\maxfigwidth}{\textwidth}
%\def\captionsize {\footnotesize}
\def\captionsize {}

\newcommand{\xsi}{{x^{(i)}}}
\newcommand{\xsd}{{x^{(d)}}}
\newcommand{\xsvcd}{{x^{(\vcd)}}}
\newcommand{\xsj}{{x^{(j)}}}
\newcommand{\ysi}{{y^{(i)}}}
\newcommand{\ysj}{{y^{(j)}}}
\newcommand{\gsi}{{\gamma^{(i)}}}
\newcommand{\wsi}{{w^{(i)}}}
\newcommand{\esi}{{\epsilon^{(i)}}}
\newcommand{\calN}{{\cal N}}
\newcommand{\calX}{{\cal X}}
\newcommand{\calY}{{\cal Y}}
\newcommand{\calL}{{\cal L}}
\newcommand{\calP}{{\cal P}}
\newcommand{\calD}{{\cal D}}
\newcommand{\ytil}{{\tilde{y}}}

\newcommand{\Ber}{{\rm Bernoulli}}
\newcommand{\E}{{\rm E}}

\newcommand{\pstar}{{p^{\ast}}}
\newcommand{\bstar}{{b^{\ast}}}
\newcommand{\dstar}{{d^{\ast}}}
\newcommand{\wstar}{{w^{\ast}}}
\newcommand{\alphastar}{\alpha^{\ast}}
\newcommand{\alphastari}{{\alpha_i^{\ast}}}
\newcommand{\betastar}{{\beta^{\ast}}}
\newcommand{\tol}{{\textit tol}}
\newcommand{\phihat}{\hat\phi}
\newcommand{\ehat}{\hat\varepsilon}
\newcommand{\hhat}{\hat{h}}
\newcommand{\hstar}{h^\ast}
\newcommand{\VC}{{\rm VC}}

\newcommand{\hwb}{{h_{w,b}}}

\begin{document}
\pagestyle{myheadings} \markboth{}{\assignmenttitle}

% <SCPD_SUBMISSION_TAG>_entire_submission

This handout includes space for every question that requires a written response.
Please feel free to use it to handwrite your solutions (legibly, please).  If
you choose to typeset your solutions, the |README.md| for this assignment includes
instructions to regenerate this handout with your typeset \LaTeX{} solutions.
\ruleskip

\LARGE
1.b
\normalsize

% <SCPD_SUBMISSION_TAG>_1b
  \begin{answer}
  % ### START CODE HERE ###
  
  The original image is storing data with 8 bits for each color so to represent a pixel it needs  3 * 8=24 bits. 
  In the compressed image, we only need 4 bits (16 colors) to represent a pixel as we have 16 clusters. \\
  So the image is compressed by factor of 6 \\
  
  % ### END CODE HERE ###
  \end{answer}
% <SCPD_SUBMISSION_TAG>_1b
\clearpage

\LARGE
2.a
\normalsize

% <SCPD_SUBMISSION_TAG>_2a
  \begin{answer}
    \begin{align*}
      \ell(\theta^{(t+1)})
      &= \alpha \ell_\text{sup}(\theta^{(t+1)}) + \ell_\text{unsup}(\theta^{(t+1)})
          &\text{Definition} \\
      &\ge \alpha \ell_\text{sup}(\theta^{(t+1)}) + \sum_{i=1}^\nexp\sum_{\zsi}Q_i^{(t)}(\zsi)\log\frac{p(\xsi,\zsi;\theta^{(t+1)})}{Q_i^{(t)}(\zsi)}
          &\text{Jensen's inequality} \\
      &\ge \\
      % ### START CODE HERE ###
     &\ge \alpha \sum_{i=1}^{\tilde{n}} \log p(\tilde{x}^{(i)},\tilde{z}^{(i)};\theta^{(t+1)})  + \sum_{i=1}^\nexp\sum_{\zsi}Q_i^{(t)}(\zsi)\log\frac{p(\xsi,\zsi;\theta^{(t+1)})}{Q_i^{(t)}(\zsi)} \\
          &\ge \alpha \sum_{i=1}^{\tilde{n}} \log p(\tilde{x}^{(i)},\tilde{z}^{(i)};\theta^{(t)})  + \sum_{i=1}^\nexp\sum_{\zsi}Q_i^{(t)}(\zsi)\log\frac{p(\xsi,\zsi;\theta^{(t)})}{Q_i^{(t)}(\zsi)} \\ 
           & = \alpha \ell_\text{sup}(\theta^{(t)}) + \ell_\text{unsup}(\theta^{(t)}) &\text{Definition} \\
             \ell_\text{semi-sup}(\theta^{(t+1)}) &\ge \ell_\text{semi-sup}(\theta^{(t)}) \\
      % ### END CODE HERE ###
    \end{align*}
  \end{answer}
% <SCPD_SUBMISSION_TAG>_2a
\clearpage

\LARGE
2.b
\normalsize

% <SCPD_SUBMISSION_TAG>_2b
  \begin{answer}
  % ### START CODE HERE ###
From Lecture notes: \\
Latent variables are  $z^{(i)} s $ meaning they are hidden/unobserved \\
E Step is given as follows: \\
$ w^{(i)}_j :=p(\zsi = j \vert \xsi ; \phi, \mu, \Sigma) $ \\

Using Baye's rule we can write this as: \\
$ p(\zsi = j \vert \xsi ; \phi, \mu, \Sigma) \\ = 
\frac{p(\xsi \vert \zsi = j ; \mu, \Sigma) p(\zsi = j; \phi)}
{\sum_{l=1}^k p(\xsi \vert \zsi = l ; \mu, \Sigma) p(\zsi=l; \phi)} $ \\

$ =
\frac
{\frac{1}{(2\pi)^{n/2}\vert\Sigma_j\vert^{1/2}} \exp(-\frac{1}{2}(\xsi - \mu_j)^T \Sigma_j^{-1}(\xsi - \mu_j))\phi_j}
{\sum_{l=1}^k \frac{1}{(2\pi)^{n/2}\vert\Sigma_l\vert^{1/2}} \exp(-\frac{1}{2}(\xsi - \mu_l)^T \Sigma_l^{-1}(\xsi - \mu_l))\phi_l} $ \\

  % ### END CODE HERE ###
  \end{answer}
% <SCPD_SUBMISSION_TAG>_2b
\clearpage

\LARGE
2.c
\normalsize

% <SCPD_SUBMISSION_TAG>_2c
  \begin{answer}
    List the parameters which need to be re-estimated in the M-step:\\\\\\

    \allowdisplaybreaks

    In order to simplify derivation, it is useful to denote $$w_j^{(i)} = Q^{(t)}_i(\zsi=j),$$ and $$\tilde{w}_j^{(i)} = \begin{cases} \alpha & \tilde{z}^{(i)} = j \\ 0 & \text{ otherwise.} \end{cases}$$

    We further denote $S = \Sigma^{-1}$, and note that because of chain rule of calculus, $\nabla_S\ell = 0 \Rightarrow \nabla_\Sigma \ell = 0$. So we choose to rewrite the M-step in terms of $S$ and maximize it w.r.t $S$, and re-express the resulting solution back in terms of $\Sigma$.

    Based on this, the M-step becomes:
    \begin{align*}
    \phi^{(t+1)}, \mu^{(t+1)}, S^{(t+1)} &=  \arg\max_{\phi,\mu,S} \sum_{i=1}^\nexp \sum_{j=1}^k Q_i^{(t)}(\zsi) \log \frac{p(\xsi,\zsi;\phi,\mu,S)}{Q_i^{(t)}(\zsi)} + \alpha \sum_{i=1}^{\tilde{\nexp}} \log p(\tilde{\xsi}, \tilde{\zsi}; \phi, \mu, S)\\
    &=\\
    % ### START CODE HERE ###
     &  \arg\max_{\phi,\mu,S} \sum_{i=1}^\nexp \sum_{j=1}^k w^{(i)}_j \log (\frac{  \frac{1}{(2\pi)^{n/2}\vert\Sigma_j\vert^{1/2}} \exp(-\frac{1}{2}(\xsi - \mu_j)^T S_j(\xsi - \mu_j))\phi_j}{w^{(i)}_j}) + \\ &  \sum_{i=1}^{\tilde{\nexp}}\sum_{j=1}^k \tilde{w}^{(i)}_j \log \frac{1}{(2\pi)^{n/2}\vert\Sigma_j\vert^{1/2}} \exp(-\frac{1}{2}(\tilde{x}^{(i)} - \mu_j)^T S_j (\tilde{x}^{(i)} - \mu_j))\phi_j \\
    % ### END CODE HERE ###
    \end{align*}

    Now, calculate the update steps by maximizing the expression within the argmax for each parameter (We will do the first for you).

    ${\mathbf \phi_j}$: We construct the Lagrangian including the constraint that $\sum_{j=1}^k \phi_j = 1$, and absorbing all irrelevant terms into constant $C$:
    \begin{align*}
    \mathcal{L}(\phi, \beta) &= C + \sum_{i=1}^\nexp\sum_{j=1}^k w^{(i)}_j \log \phi_j + \sum_{i=1}^{\tilde{\nexp}}\sum_{j=1}^k \tilde{w}^{(i)}_j \log \phi_j + \beta\left(\sum_{j=1}^k \phi_j - 1\right) \\
    \nabla_{\phi_j}\mathcal{L}(\phi, \beta) &=  \sum_{i=1}^\nexp w^{(i)}_j\frac{1}{\phi_j} + \sum_{i=1}^{\tilde{\nexp}} \tilde{w}^{(i)}_j\frac{1}{\phi_j} + \beta = 0 \\
    &\Rightarrow \phi_j = \frac{\sum_{i=1}^\nexp w^{(i)}_j + \sum_{i=1}^{\tilde{\nexp}} \tilde{w}^{(i)}_j}{-\beta} \\
    \nabla_\beta\mathcal{L}(\phi,\beta) &= \sum_{j=1}^k \phi_j -1 = 0 \\
    &\Rightarrow \sum_{j=1}^k \frac{\sum_{i=1}^\nexp w^{(i)}_j + \sum_{i=1}^{\tilde{\nexp}} \tilde{w}^{(i)}_j}{-\beta} = 1 \\
    &\Rightarrow -\beta = \sum_{j=1}^k \left(\sum_{i=1}^\nexp w^{(i)}_j + \sum_{i=1}^{\tilde{\nexp}} \tilde{w}^{(i)}_j\right)  \\
    \Rightarrow \phi_j^{(t+1)} &= \frac{ \sum_{i=1}^\nexp w_j^{(i)} + \sum_{i=1}^{\tilde{\nexp}}\tilde{w}_j^{(i)}} { \sum_{j=1}^k \left(\sum_{i=1}^\nexp w^{(i)}_j + \sum_{i=1}^{\tilde{\nexp}} \tilde{w}^{(i)}_j\right) } \\
    &= \frac{ \sum_{i=1}^\nexp w_j^{(i)} + \sum_{i=1}^{\tilde{\nexp}}\tilde{w}_j^{(i)}} { \nexp + \alpha \tilde{\nexp}}
    \end{align*}

    ${\mathbf \mu_j}$: Next, derive the update for $\mu_j$.  Do this by maximizing the expression with the argmax above with respect to $\mu_j$.\\

    First, calculate the gradient with respect to $\mu_j$:

    \begin{flalign*}
    \nabla_{\mu_j} &=
    % ### START CODE HERE ###
 \sum_{i=1}^\nexp  w^{(i)}_j (S_j)(\xsi - \mu_j) + \sum_{i=1}^{\tilde{\nexp}} \tilde{w}^{(i)}_j (S_j) (\tilde{x}^{(i)} - \mu_j))\\
    % ### END CODE HERE ###
    \end{flalign*}

    Next, set the gradient to zero and solve for $\mu_j$:

    \begin{align*}
    0 &= \\
    % ### START CODE HERE ###
    & \sum_{i=1}^\nexp  w^{(i)}_j (S_j)(\xsi - \mu_j) + \sum_{i=1}^{\tilde{\nexp}} \tilde{w}^{(i)}_j (S_j) (\tilde{x}^{(i)} - \mu_j))\\
     & \sum_{i=1}^\nexp  w^{(i)}_j (S_j)(\xsi - \mu_j) + \sum_{i=1}^{\tilde{\nexp}} \tilde{w}^{(i)}_j (S_j) (\tilde{x}^{(i)} - \mu_j))\\
     & \mu_j = \frac{\sum_{i=1}^\nexp  w^{(i)}_j\xsi + \sum_{i=1}^{\tilde{\nexp}}\tilde{w}^{(i)}_j\tilde{x}^{(i)}} {\sum_{i=1}^\nexp  w^{(i)}_j\ + \sum_{i=1}^{\tilde{\nexp}}\tilde{w}^{(i)}_j}
    % ### END CODE HERE ###
    \end{align*}

    ${\mathbf \Sigma_j}$: Finally, derive the update for $\Sigma_j$ via $S_j$.  Again, Do this by maximizing the expression with the argmax above with respect to $S_j$.\\.

    First, calculate the gradient with respect to $S_j$:

    \begin{flalign*}
    \nabla_{S_j} &= 
    % ### START CODE HERE ###
     \nabla_{S_j} (\sum_{i=1}^\nexp \sum_{j=1}^k w^{(i)}_j \log (\frac{  \frac{1}{(2\pi)^{n/2}\vert\Sigma_j\vert^{1/2}} \exp(-\frac{1}{2}(\xsi - \mu_j)^T S_j(\xsi - \mu_j))\phi_j}{w^{(i)}_j})) + \\ &  \nabla_{S_j}(\sum_{i=1}^{\tilde{\nexp}}\sum_{j=1}^k \tilde{w}^{(i)}_j \log \frac{1}{(2\pi)^{n/2}\vert\Sigma_j\vert^{1/2}} \exp(-\frac{1}{2}(\tilde{x}^{(i)} - \mu_j)^T S_j (\tilde{x}^{(i)} - \mu_j))\phi_j) \\ &=
    \sum_{i=1}^\nexp w^{(i)}_j  (-\frac{1}{2}(S_j) +\frac{1}{2} (\xsi - \mu_j)^T (\xsi - \mu_j))(S_j^{-1}) + \\ &  \sum_{i=1}^{\tilde{\nexp}} \tilde{w}^{(i)}_j (-\frac{1}{2}(S_j) +\frac{1}{2} (\tilde{x}^{(i)} - \mu_j)^T (\tilde{x}^{(i)} - \mu_j))(S_j^{-1})) \\
    % ### END CODE HERE ###
    \end{flalign*}

    Next, set the gradient to zero and solve for $S_j$:

    \begin{align*}
    0 &= \\
    % ### START CODE HERE ###
     \sum_{i=1}^\nexp w^{(i)}_j  (-\frac{1}{2}(S_j) +\frac{1}{2} (\xsi - \mu_j)^T (\xsi - \mu_j))(S_j^{-1}) + \\    \sum_{i=1}^{\tilde{\nexp}} \tilde{w}^{(i)}_j (-\frac{1}{2}(S_j) +\frac{1}{2} (\tilde{x}^{(i)} - \mu_j)^T (\tilde{x}^{(i)} - \mu_j))(S_j^{-1})) \\ 
     \Sigma_j =\frac{ \sum_{i=1}^\nexp w^{(i)}_j(\xsi - \mu_j) (\xsi - \mu_j)^T + \sum_{i=1}^{\tilde{\nexp}} \tilde{w}^{(i)}_j (\tilde{x}^{(i)} - \mu_j) (\tilde{x}^{(i)} - \mu_j)^T} {\sum_{i=1}^\nexp  w^{(i)}_j\ + \sum_{i=1}^{\tilde{\nexp}}\tilde{w}^{(i)}_j} \\
    % ### END CODE HERE ###
    \end{align*}

    This results in the final set of update expressions:
    \begin{align*}
      \phi_j & := \\
      % ### START CODE HERE ###
      & \frac{ \sum_{i=1}^\nexp w_j^{(i)} + \sum_{i=1}^{\tilde{\nexp}}\tilde{w}_j^{(i)}} { \nexp + \alpha \tilde{\nexp}} \\
      % ### END CODE HERE ###
      \mu_j & :=  \\
      % ### START CODE HERE ###
      & \frac{\sum_{i=1}^\nexp  w^{(i)}_j\xsi + \sum_{i=1}^{\tilde{\nexp}}\tilde{w}^{(i)}_j\tilde{x}^{(i)}} {\sum_{i=1}^\nexp  w^{(i)}_j\ + \sum_{i=1}^{\tilde{\nexp}}\tilde{w}^{(i)}_j} \\
      % ### END CODE HERE ###
      \Sigma_j & :=  \\
      % ### START CODE HERE ###
      & \frac{ \sum_{i=1}^\nexp w^{(i)}_j(\xsi - \mu_j) (\xsi - \mu_j)^T + \sum_{i=1}^{\tilde{\nexp}} \tilde{w}^{(i)}_j (\tilde{x}^{(i)} - \mu_j) (\tilde{x}^{(i)} - \mu_j)^T} {\sum_{i=1}^\nexp  w^{(i)}_j\ + \sum_{i=1}^{\tilde{\nexp}}\tilde{w}^{(i)}_j}
      % ### END CODE HERE ###
    \end{align*}
  \end{answer}
% <SCPD_SUBMISSION_TAG>_2c
\clearpage

\LARGE
2.f
\normalsize

% <SCPD_SUBMISSION_TAG>_2f
  \begin{answer}
    % ### START CODE HERE ###
    i. \\
    Unsupervised EM took a lot more iterations to converge as compared to Semi-Supervised EM. \\ 
    Unsupervised EM took almost 1000 of iteration to converge. \\
    Semi-Supervised EM took approximately 50-60 iterations. \\

ii. \\
The assignments by unsupervised EM were random with different random initializations. \\
The assignments by semi-supervised EM were same or roughly the same.  \\ 
Semi-supervised EM are more stable than unsupervised EM. \\
\\

iii. \\
The pictures of semi-supervised EM have nearly 3 same low-variance Gaussian distributions, and 1 high-variance Gaussian distribution. \\
The pictures of unsupervised EM have  four Gaussian distributions with different variances. \\
The overall quality of assignments by semi-supervised EM are higher than unsupervised EM. \\

    % ### END CODE HERE ###
  \end{answer}
% <SCPD_SUBMISSION_TAG>_2f
\clearpage

% <SCPD_SUBMISSION_TAG>_entire_submission

\end{document}
