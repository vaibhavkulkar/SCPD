\item \points{1b}
In this question you are going to implement a Naive Bayes classifier for spam
classification with {\bf multinomial event model} and Laplace smoothing.

Code your implementation by completing the |fit_naive_bayes_model()|
and |predict_from_naive_bayes_model()| functions in
|src-spam/submission.py|.

Now the functions in |src-spam/submission.py| should be able to train a Naive Bayes model.  Use autograder test case |1b-1-basic| to
compute your prediction accuracy and then save your resulting predictions
to\\|spam_naive_bayes_predictions_(soln)|.


{\bf Remark.} If you implement Naive Bayes the straightforward way, you will find
that the computed $p(x\vert y) = \prod_i p(x_i \vert  y)$ often equals zero.  This is
because $p(x\vert y)$, which is the product of many numbers less than one, is a very
small  number. The standard computer representation of real numbers cannot
handle numbers that are too small, and instead rounds them off to zero.  (This
is called  ``underflow.'')  You'll have to find a way to compute Naive Bayes'
predicted  class labels without explicitly representing very small numbers such
as $p(x\vert y)$.
[\textbf{Hint:} Think about using logarithms.]
