% This contents of this file will be inserted into the _Solutions version of the
% output tex document.  Here's an example:

% If assignment with subquestion (1.a) requires a written response, you will
% find the following flag within this document: <SCPD_SUBMISSION_TAG>_1a
% In this example, you would insert the LaTeX for your solution to (1.a) between
% the <SCPD_SUBMISSION_TAG>_1a flags.  If you also constrain your answer between the
% START_CODE_HERE and END_CODE_HERE flags, your LaTeX will be styled as a
% solution within the final document.

% Please do not use the '<SCPD_SUBMISSION_TAG>' character anywhere within your code.  As expected,
% that will confuse the regular expressions we use to identify your solution.
\def\assignmentnum{1 }
\def\assignmenttitle{XCS229i Problem Set \assignmentnum}
\newcommand{\newsec}{\section}
\newcommand{\denselist}{\itemsep 0pt\partopsep 0pt}
\newcommand{\bitem}{\begin{itemize}\denselist}
\newcommand{\eitem}{\end{itemize}}
\newcommand{\benum}{\begin{enumerate}\denselist}
\newcommand{\eenum}{\end{enumerate}}

\newcommand{\fig}[1]{\private{\begin{center}
{\Large\bf ({#1})}
\end{center}}}

\newcommand{\cpsf}[1]{{\centerline{\psfig{#1}}}}
\newcommand{\mytitle}[1]{\centerline{\LARGE\bf #1}}

\newcommand{\myw}{{\bf w}}

\newcommand{\mypar}[1]{\vspace{1ex}\noindent{\bf {#1}}}

\def\thmcolon{\hspace{-.85em} {\bf :} }

\newtheorem{THEOREM}{Theorem}[section]
\newenvironment{theorem}{\begin{THEOREM} \thmcolon }%
                        {\end{THEOREM}}
\newtheorem{LEMMA}[THEOREM]{Lemma}
\newenvironment{lemma}{\begin{LEMMA} \thmcolon }%
                      {\end{LEMMA}}
\newtheorem{COROLLARY}[THEOREM]{Corollary}
\newenvironment{corollary}{\begin{COROLLARY} \thmcolon }%
                          {\end{COROLLARY}}
\newtheorem{PROPOSITION}[THEOREM]{Proposition}
\newenvironment{proposition}{\begin{PROPOSITION} \thmcolon }%
                            {\end{PROPOSITION}}
\newtheorem{DEFINITION}[THEOREM]{Definition}
\newenvironment{definition}{\begin{DEFINITION} \thmcolon \rm}%
                            {\end{DEFINITION}}
\newtheorem{CLAIM}[THEOREM]{Claim}
\newenvironment{claim}{\begin{CLAIM} \thmcolon \rm}%
                            {\end{CLAIM}}
\newtheorem{EXAMPLE}[THEOREM]{Example}
\newenvironment{example}{\begin{EXAMPLE} \thmcolon \rm}%
                            {\end{EXAMPLE}}
\newtheorem{REMARK}[THEOREM]{Remark}
\newenvironment{remark}{\begin{REMARK} \thmcolon \rm}%
                            {\end{REMARK}}
%\newenvironment{proof}{\noindent {\bf Proof:} \hspace{.677em}}%
%                      {}

%theorem
\newcommand{\thm}{\begin{theorem}}
%lemma
\newcommand{\lem}{\begin{lemma}}
%proposition
\newcommand{\pro}{\begin{proposition}}
%definition
\newcommand{\dfn}{\begin{definition}}
%remark
\newcommand{\rem}{\begin{remark}}
%example
\newcommand{\xam}{\begin{example}}
%corollary
\newcommand{\cor}{\begin{corollary}}
%proof
\newcommand{\prf}{\noindent{\bf Proof:} }
%end theorem
\newcommand{\ethm}{\end{theorem}}
%end lemma
\newcommand{\elem}{\end{lemma}}
%end proposition
\newcommand{\epro}{\end{proposition}}
%end definition
\newcommand{\edfn}{\bbox\end{definition}}
%end remark
\newcommand{\erem}{\bbox\end{remark}}
%end example
\newcommand{\exam}{\bbox\end{example}}
%end corollary
\newcommand{\ecor}{\end{corollary}}
%end proof
\newcommand{\eprf}{\bbox\vspace{0.1in}}
%begin equation
\newcommand{\beqn}{\begin{equation}}
%end equation
\newcommand{\eeqn}{\end{equation}}

%\newcommand{\eqref}[1]{Eq.~\ref{#1}}

\newcommand{\KB}{\mbox{\it KB\/}}
\newcommand{\infers}{\vdash}
\newcommand{\sat}{\models}
\newcommand{\bbox}{\vrule height7pt width4pt depth1pt}

\newcommand{\act}[1]{\stackrel{{#1}}{\rightarrow}}
\newcommand{\at}[1]{^{(#1)}}

\newcommand{\argmax}{{\rm argmax}}

\newcommand{\rimp}{\Rightarrow}
\newcommand{\dimp}{\Leftrightarrow}

\newcommand{\bX}{\mbox{\boldmath $X$}}
\newcommand{\bY}{\mbox{\boldmath $Y$}}
\newcommand{\bZ}{\mbox{\boldmath $Z$}}
\newcommand{\bU}{\mbox{\boldmath $U$}}
\newcommand{\bE}{\mbox{\boldmath $E$}}
\newcommand{\bx}{\mbox{\boldmath $x$}}
\newcommand{\be}{\mbox{\boldmath $e$}}
\newcommand{\by}{\mbox{\boldmath $y$}}
\newcommand{\bz}{\mbox{\boldmath $z$}}
\newcommand{\bu}{\mbox{\boldmath $u$}}
\newcommand{\bd}{\mbox{\boldmath $d$}}
\newcommand{\smbx}{\mbox{\boldmath $\scriptstyle x$}}
\newcommand{\smbd}{\mbox{\boldmath $\scriptstyle d$}}
\newcommand{\smby}{\mbox{\boldmath $\scriptstyle y$}}
\newcommand{\smbe}{\mbox{\boldmath $\scriptstyle e$}}

\newcommand{\Parents}{\mbox{\it Parents\/}}
\newcommand{\B}{{\cal B}}
\newcommand{\calH}{{\cal H}}

\newcommand{\word}[1]{\mbox{\it #1\/}}
\newcommand{\Action}{\word{Action}}
\newcommand{\Proposition}{\word{Proposition}}
\newcommand{\true}{\word{true}}
\newcommand{\false}{\word{false}}
\newcommand{\Pre}{\word{Pre}}
\newcommand{\Add}{\word{Add}}
\newcommand{\Del}{\word{Del}}
\newcommand{\Result}{\word{Result}}
\newcommand{\Regress}{\word{Regress}}
\newcommand{\Maintain}{\word{Maintain}}

\newcommand{\bor}{\bigvee}
\newcommand{\invert}[1]{{#1}^{-1}}

\newcommand{\commentout}[1]{}

\newcommand{\bmu}{\mbox{\boldmath $\mu$}}
\newcommand{\btheta}{\mbox{\boldmath $\theta$}}
\newcommand{\IR}{\mbox{$I\!\!R$}}

\newcommand{\tval}[1]{{#1}^{1}}
\newcommand{\fval}[1]{{#1}^{0}}

\newcommand{\tr}{{\rm tr}}
\newcommand{\vecy}{{\vec{y}}}
\renewcommand{\Re}{{\mathbb R}}

\def\twofigbox#1#2{%
\noindent\begin{minipage}{\textwidth}%
\epsfxsize=0.35\maxfigwidth
\noindent \epsffile{#1}\hfill
\epsfxsize=0.35\maxfigwidth
\epsffile{#2}\\
\makebox[0.35\textwidth]{(a)}\hfill\makebox[0.35\textwidth]{(b)}%
\end{minipage}}

\def\twofigboxcd#1#2{%
\noindent\begin{minipage}{\textwidth}%
\epsfxsize=0.35\maxfigwidth
\noindent \epsffile{#1}\hfill
\epsfxsize=0.35\maxfigwidth
\epsffile{#2}\\
\makebox[0.35\textwidth]{(c)}\hfill\makebox[0.35\textwidth]{(d)}%
\end{minipage}}

\def\twofigboxnolabel#1#2{%
\begin{minipage}{\textwidth}%
\epsfxsize=0.35\maxfigwidth
\noindent \epsffile{#1}\hfill
\epsfxsize=0.35\maxfigwidth
\epsffile{#2}\\
%\makebox[0.48\textwidth]{(a)}\hfill\makebox[0.48\textwidth]{(b)}%
\end{minipage}
}

\def\twofigboxnolabelFive#1#2{%
\begin{minipage}{\textwidth}%
\hbox to 0.5in{}\epsfxsize=0.35\maxfigwidth
\noindent \epsffile{#1}\hfill
\epsfxsize=0.35\maxfigwidth
\epsffile{#2}\hbox to 0.5in{}\\
%\makebox[0.48\textwidth]{(a)}\hfill\makebox[0.48\textwidth]{(b)}%
\end{minipage}
}

\def\threefigbox#1#2#3{%
\noindent\begin{minipage}{\textwidth}%
\epsfxsize=0.33\maxfigwidth
\noindent \epsffile{#1}\hfill
\epsfxsize=0.33\maxfigwidth
\noindent \epsffile{#2}\hfill 
\epsfxsize=0.33\maxfigwidth
\epsffile{#3}\\
\makebox[0.31\textwidth]{{\scriptsize (a)}}\hfill%
\makebox[0.31\textwidth]{{\scriptsize (b)}}\hfill
\makebox[0.31\textwidth]{{\scriptsize (c)}}%
\smallskip
\end{minipage}}

\def\threefigboxnolabel#1#2#3{%
\noindent\begin{minipage}{\textwidth}%
\epsfxsize=0.33\maxfigwidth
\noindent \epsffile{#1}\hfill
\epsfxsize=0.33\maxfigwidth
\noindent \epsffile{#2}\hfill 
\epsfxsize=0.33\maxfigwidth
\epsffile{#3}\\
%\makebox[0.31\textwidth]{{\scriptsize (a)}}\hfill%
%\makebox[0.31\textwidth]{{\scriptsize (b)}}\hfill
%\makebox[0.31\textwidth]{{\scriptsize (c)}}%
%\smallskip
\end{minipage}}

\newlength{\maxfigwidth}
\setlength{\maxfigwidth}{\textwidth}
%\def\captionsize {\footnotesize}
\def\captionsize {}

\newcommand{\xsi}{{x^{(i)}}}
\newcommand{\ssi}{{s^{(i)}}}
\newcommand{\xsd}{{x^{(d)}}}
\newcommand{\xsj}{{x^{(j)}}}
\newcommand{\ysi}{{y^{(i)}}}
\newcommand{\ysj}{{y^{(j)}}}
\newcommand{\gsi}{{\gamma^{(i)}}}
\newcommand{\wsi}{{w^{(i)}}}
\newcommand{\esi}{{\epsilon^{(i)}}}
\newcommand{\calN}{{\cal N}}
\newcommand{\calX}{{\cal X}}
\newcommand{\calY}{{\cal Y}}
\newcommand{\calL}{{\cal L}}
\newcommand{\calP}{{\cal P}}
\newcommand{\calD}{{\cal D}}
\newcommand{\ytil}{{\tilde{y}}}

\newcommand{\Ber}{{\rm Bernoulli}}
\newcommand{\E}{\mathbb{E}}

\newcommand{\pstar}{{p^{\ast}}}
\newcommand{\bstar}{{b^{\ast}}}
\newcommand{\dstar}{{d^{\ast}}}
\newcommand{\wstar}{{w^{\ast}}}
\newcommand{\alphastar}{\alpha^{\ast}}
\newcommand{\alphastari}{{\alpha_i^{\ast}}}
\newcommand{\betastar}{{\beta^{\ast}}}
\newcommand{\tol}{{\textit tol}}
\newcommand{\phihat}{\hat\phi}
\newcommand{\ehat}{\hat\varepsilon}
\newcommand{\hhat}{\hat{h}}
\newcommand{\hstar}{h^\ast}
\newcommand{\VC}{{\rm VC}}

\newcommand{\hwb}{{h_{w,b}}}

\begin{document}
\pagestyle{myheadings} \markboth{}{\assignmenttitle}

% <SCPD_SUBMISSION_TAG>_entire_submission

This handout includes space for every question that requires a written response.
Please feel free to use it to handwrite your solutions (legibly, please).  If
you choose to typeset your solutions, the |README.md| for this assignment includes
instructions to regenerate this handout with your typeset \LaTeX{} solutions.
\ruleskip

\LARGE
1.a
\normalsize


% <SCPD_SUBMISSION_TAG>_1a
\begin{answer}
  % ### START CODE HERE ###
\LARGE
Lets start with the definition of that integral of density function is equal to 1 over the entire space:
\\
$ \int p (y; \eta)dy = 1 $
\\
Applying same to exponential family distribution and trying to find out $ a (\eta) $:
\\ 
$ \int p(y; \eta) dy = \int  b(y)\exp(\eta y - a(\eta)) dy = 1 $ i.e
\\  
$  \int  b(y)\exp(\eta y - a(\eta)) dy = 1 $
\\ We can rewrite this as: \\
$ \exp(-a(\eta)) \int b(y)\exp(\eta y) dy = 1 $
\\ 
$ \exp(a(\eta)) = \int b(y)\exp(\eta y) dy $ 
\\ 
$ a(\eta) = \log  \int b(y)\exp(\eta y) dy $ 
\\ Lets take derivative of $a(\eta)$ with respect to $\eta$ and apply hint provided in 1.a
\\ 
$ \frac {\partial a(\eta)}{ \partial \eta}  = \frac{\partial}{\partial (\eta) } \log  \int b(y)\exp(\eta y) dy $
\\
$= \frac{\int yb(y) \exp (\eta y) dy} {\int b(y) \exp (\eta y) dy} $
\\ As per pervious steps we can replace denominator with $ \exp a(\eta) $ 
\\
$=  \frac{\int yb(y) \exp (\eta y) dy} { \exp a(\eta)} $
\\
$=  \int  y b(y)\exp(\eta y - a(\eta)) dy = E [Y;\eta] $
\\ \\ This proves that the first derivate of $a(\eta$) w.r.t $\eta$ is equivalent to the mean of exponentail family distribution

  % ### END CODE HERE ###
\end{answer}
% <SCPD_SUBMISSION_TAG>_1a
\clearpage

\LARGE
1.b
\normalsize

% <SCPD_SUBMISSION_TAG>_1b
\begin{answer}
  % ### START CODE HERE ###
\LARGE
Lets startby computing second derivative of $a(\eta)$ w.r.t $\eta$ using defintion computed in previous answer 1.a
\\
$ \frac {\partial^2 a(\eta)}{ \partial \eta^2}  = \frac{\partial}{\partial (\eta) }   \int  y b(y)\exp(\eta y - a(\eta)) dy $
\\   \\ 
$ =  \int  y b(y)\exp(\eta y - a(\eta)) (y-a^\prime(\eta)) dy$
\\   \\ 
$ =  \int p(y;\eta)y^2dy -a^\prime(\eta) \int p(y;\eta) ydy $
\\   \\ 
$ = E[Y^2;\eta] - E[Y;\eta] E[Y;\eta] $
\\   \\ 
$ = Var[Y;\eta] $ \\ \\
This shows that the variance of an exponential family distribution is the second derivative of the log-partition function w.r.t. the natural parameter.

  % ### END CODE HERE ###
\end{answer}
% <SCPD_SUBMISSION_TAG>_1b
\clearpage

\LARGE
1.c
\normalsize

% <SCPD_SUBMISSION_TAG>_1c
\begin{answer}
  % ### START CODE HERE ###
  \LARGE
  Lets start with definition of negative log likelyhood
  $ NLL = -log(p(y;\eta))$ \\
  $ = -log(b(y) \exp (\eta y- a(\eta))) $ \\
  $ = -(log(b(y)) + log(\exp(\eta y- a(\eta)))) $ \\ This can be rewritten as \\
  $ = -(log(b(y)) + (\eta y- a(\eta))) $ \\
  $ = -(log(b(y)) + (\theta^{T}xy- a(\theta^{T}x))) $ \\ \\ 
  Now lets take hessian of the NLL wrt to $\theta$ : \\
  $ \nabla^{2}_\theta(NLL) = \nabla^{2}_\theta ( -(log(b(y)) + (\theta^{T}xy- a(\theta^{T}x)))  ) $ \\
  $ = \nabla^{2}_\theta ( -(\theta^{T}xy- a(\theta^{T}x))) $ \\ 
  The second order derivative of $  -(\theta^{T}xy)$ w.r.t $\theta $ is equal to 0, so:\\
  $ =  \nabla^{2}_\theta (a(\theta^{T}x)) $ \\ 
  $ = Var(Y;\eta)$ \\
  As variance of any probability distribution is non negative and therefore the Hessian of GLM’s NLL loss is PSD, and hence convex.
  
  % ### END CODE HERE ###
\end{answer}
% <SCPD_SUBMISSION_TAG>_1c
\clearpage

\LARGE
2.a
\normalsize

% <SCPD_SUBMISSION_TAG>_2a
\begin{answer}
  \begin{align*}
      J(\theta)&=\\
      % ### START CODE HERE ###
\frac{1}{2} \sum_{i=1}^{\nexp} (h_\theta(\hat{x}^{(i)} - \ysi)^2.      
      % ### END CODE HERE ###
  \end{align*}

  Differentiating this objective, we get:
  \begin{align*}
      \nabla_{\theta} J(\theta)&=\\
      % ### START CODE HERE ###
      \frac{\partial}{\partial \theta_{j}}\frac{1}{2}  (h_\theta(\hat{x}) - y)^2 = \\
      (h_\theta(\hat{x}) - y)x_{j}
      % ### END CODE HERE ###
  \end{align*}
  % ### START CODE HERE ###
  % ### END CODE HERE ###
  The gradient descent update rule is
  %
  \begin{equation*}
  \theta := \theta - \lambda \nabla_{\theta} J(\theta)
  \end{equation*}
  %
  which reduces here to:
  % ### START CODE HERE ###
  \\
$  \theta := \theta - \lambda (h_\theta(\hat{x}) - y)x_{j} $
\\ Rearranging terms and in general for i \\
$  \theta := \theta + \lambda (y^{(i)} - h_\theta(\hat{x}^{(i)}))\hat{x}_{j}^{(i)} $
  % ### END CODE HERE ###
\end{answer}
% <SCPD_SUBMISSION_TAG>_2a
\clearpage

\LARGE
2.d
\normalsize

% <SCPD_SUBMISSION_TAG>_2d
\begin{answer}
  % ### START CODE HERE ###
  \Large
  For k=1 (or 2) the fit is almost a straight line\\
  For k=3 the fit starts to show the sin wave pattern \\
  For k=5,10 the fit is more natural to data points and is closer also\\
  For k=20 the curve passes through most of the points and also start showing signs of overfitting as we can see some curvatures beyond the given point.\\
So as k increases fit is passing through more and more points and tends to overfitting.
  
  % ### END CODE HERE ###
\end{answer}
% <SCPD_SUBMISSION_TAG>_2d
\clearpage

\LARGE
2.f
\normalsize

% <SCPD_SUBMISSION_TAG>_2f
\begin{answer}
  % ### START CODE HERE ###
  \Large
  Compared to 2.c  we can see the fitted model taking a sin wave pattern. \\
  This is even true for low values of k like 1 or 2. \\
  The reason for this is that the training data is also created using sin function. \\
  After adding a sin(x) to the polynomial regression even for low value of x we get good fit as compared to 2.c \\
  % ### END CODE HERE ###
\end{answer}
% <SCPD_SUBMISSION_TAG>_2f
\clearpage

\LARGE
2.h
\normalsize

% <SCPD_SUBMISSION_TAG>_2h
\begin{answer}
  % ### START CODE HERE ###
  \Large
  As the training dataset is small the fitting of the training dataset changes with K as follows:\\ \\ 
  For polynomial regression, \\
  For lower values of K (= 1 or 2) the fit is not going over any data point\\
  For K (=3,5) fit is closer to training data point or passes through some of the training data points and is more natural \\
  But as K increases(10,20) we can see overfitting i.e the long curves in sin wave\\  \\  
  For polynomial and sinusoidal features, \\ 
  Fitting of the data is more natural even with lower values of K like 1,2,3,5 \\
  But as K increases we can see overfitting
  % ### END CODE HERE ###
\end{answer}
% <SCPD_SUBMISSION_TAG>_2h
\clearpage

% <SCPD_SUBMISSION_TAG>_entire_submission

\end{document}
